%% Copyright (C) 2021 Alessandro Clerici Lorenzini
%
% This work may be distributed and/or modified under the
% conditions of the LaTeX Project Public License, either version 1.3
% of this license or (at your option) any later version.
% The latest version of this license is in
%   http://www.latex-project.org/lppl.txt
% and version 1.3 or later is part of all distributions of LaTeX
% version 2005/12/01 or later.
%
% This work has the LPPL maintenance status `maintained'.
%
% The Current Maintainer of this work is Alessandro Clerici Lorenzini
%
% This work consists of the files listed in work.txt


\section{Variabili aleatorie discrete}
\begin{defin}
	In uno spazio di probabilità $(\Omega, \A, p)$, una variabile aleatoria (o casuale) è una funzione\footnote{in verità, la funzione deve soddisfare il criterio: $\{\omega \mid X(\omega)\leq r\}\in\A\quad\forall r\in\R$} $X:\Omega\to \R$.
    L'immagine di una variabile aleatoria si dice supporto e si indica con $D_X$. Ogni elemento del supporto, cioè l'immagine per $X$ di un elemento di $\Omega$, si dice specificazione. Una specificazione generica di $X$ si indica tipicamente con $x$.
\end{defin}


\subsection{Definizioni}
\begin{defin}
	Una variabile aleatoria è discreta se e solo se il suo supporto è numerabile.
\end{defin}

\begin{examp}
	Se $\Omega$ è l'insieme di esiti del lancio di due dadi, una possibile variabile aleatoria $X$ è la somma degli esiti dei due dadi. In questo caso immagini di $X$ non sono equiprobabili. Se invece per $X$ si sceglie il valore del primo dado, le immagini sono equiprobabili.
\end{examp}

\begin{defin}[funzione indicatrice]
	Siano $A$ e $D$ insiemi tali che $A\subseteq D$. La funzione indicatrice (o caratteristica) di $A$ rispetto a $D$ è la funzione $I:D\to \{0,1\}$ che associa a un elemento di $D$ il numero $1$ se l'elemento appartiene ad $A$, o $0$ altrimenti:
	\begin{equation*}
		I_A(x) := \begin{cases}
			1 \qquad x\in A \\
			0 \qquad x\notin A
		\end{cases}
	\end{equation*}
\end{defin}

\begin{defin}[funzione indicatrice di un evento] \label{def:findiceven}
	La funzione indicatrice di un evento $A$ è la variabile aleatoria che ha specificazione $1$ se l'evento $A$ si verifica e $0$ se l'evento non si verifica.
\end{defin}



\subsection{Funzione di ripartizione}
\begin{defin}[funzione di ripartizione] \label{def:fripar}
	Data una variabile aleatoria $X$, la funzione di ripartizione (o di distribuzione cumulativa) $F_X:\R\to[0,1]$ è la funzione che associa a un valore $x\in\R$ la probabilità che l'esito di $X$ ne sia minore o uguale:
	\begin{equation*}
		\forall x\in\R\quad F_X(x):=P(X\leq x)
	\end{equation*}
\end{defin}

La funzione di ripartizione si annulla quandunque l'input è minore della minima specificazione.




\subsection{Funzione di massa di probabilità}
\begin{defin}[funzione di massa di probabilità] \label{def:fmassaprob}
	Data una variabile aleatoria discreta $X$, la funzione di massa di probabilità $p_X:\R\to[0,1]$ è la funzione che associa a un valore $x\in\R$ la probabilità che l'esito di $X$ sia uguale a $x$:
	\begin{equation*}
		\forall x \in\R\quad p_X(x)= P(x=X)
	\end{equation*}
\end{defin}

La somma delle immagini della funzione di massa di probabilità per tutte le specificazioni di $X$ è ovviamente uguale a $1$:
\begin{equation} \label{eq:sommamassa}
	\sum_i p_X (x_i) = 1
\end{equation}

La funzione di massa di probabilità di $X$ si annulla nei punti di $\R$ non appartenenti al supporto di $X$.

\begin{prop}
	Per le variabili aleatorie discrete vale la seguente relazione tra la funzione di ripartizione e la funzione di massa di probabilità:
	\begin{equation*}
		F_X(x)=\sum_{a\leq x} p_X(a)
	\end{equation*}
	con $a\in D_X$.
\end{prop}
\begin{proof}
	\begin{align*}
		F_X(x) & = P(X\leq x)                             \bc{definizione \ref{def:fripar}}     \\
		       & = P\left(\bigcup_{a\leq x}\{X=a\}\right)                                       \\
		       & = \sum_{a\leq x} P(X=a)                  \bc{unione di eventi disgiunti}       \\
		       & = \sum_{a\leq x} p_X(a)                  \bc{definizione \ref{def:fmassaprob}}
	\end{align*}
\end{proof}



\subsection{Valore atteso}
Il valore atteso di una variabile aleatoria $X$ è un indice di centralità delle specificazioni della variabile aleatoria.
\begin{defin}[valore atteso] \label{def:valatt}
	Data una variabile aleatoria discreta $X$, il valore atteso di $X$ è la media delle specificazioni di $X$ pesata con le rispettive probabilità:
	\begin{equation*}
		\ev{X}=\sum_i x_i P(X=x_i)
	\end{equation*}
	Talvolta si usa anche la notazione $\mu=\mu_X=\ev{X}$.
\end{defin}
\noindent
In generale, non è detto che la sommatoria che definisce il valore atteso converga.

\begin{prop} \label{prop:indvalatt}
	Il valore atteso della funzione indicatrice di un evento è uguale alla probabilità dell'evento:
	\begin{equation*}
		\ev{I_A}=P(A)
	\end{equation*}
\end{prop}
\begin{proof}
	Applicando la definizione \ref{def:valatt} di valore atteso e la \ref{def:findiceven} di funzione indicatrice di un evento:
	\begin{equation*}
		\ev{I_A}=1*P(A)+0*(1-P(A))=P(A)
	\end{equation*}
\end{proof}

\subsubsection{Linearità del valore atteso}
\begin{prop}
	Il valore atteso di una variabile aleatoria discreta $X$ è lineare rispetto a una trasformazione lineare $ax+b$ applicata alle specificazioni $x_i$ di $X$:
	\begin{equation*}
		D_Y = \{ax_i+b \mid x_i\in D_X\} \qquad\Rightarrow\qquad \ev{Y}=a\ev{X}+b
	\end{equation*}
	Si scrive anche $Y=aX+b$.
\end{prop}
\begin{proof}
	Poiché le probabilità per le rispettive specificazioni rimangono invariate:
	\begin{align*}
		\ev{Y} & = \sum_i (a x_i + b) P(X=x_i)                                                    \\
		       & = a\sum_i x_i P(X=x_i) + b\sum_i P(X=x_i)                                        \\
		       & = a\underbrace{\sum_i x_i p_X(x_i)}_{\ev{x}} + b\underbrace{\sum_i p_X(x_i)}_{1} \\
		       & = a \ev{x} + b \bc{\eqref{eq:sommamassa} e definizione \ref{def:valatt}}
	\end{align*}
\end{proof}



\subsection{Varianza di una variabile aleatoria}
\begin{defin}[varianza di una variabile aleatoria] \label{def:var}
	La varianza di una variabile aleatoria $X$ è definita come segue:
	\begin{equation*}
		\var(X) = \sigma^2_X := \ev{(X-\ev{X})^2}
	\end{equation*}
\end{defin}

\begin{defin}[deviazione standard di una variabile aleatoria]
	La deviazione standard di una variabile aleatoria $X$ è così definita:
	\begin{equation*}
		\sigma(X) = \sigma_X :=\sqrt{\var(X)}
	\end{equation*}
\end{defin}


\subsubsection{Proprietà della varianza}
\begin{prop} \label{prop:varalt}
	La varianza di $X$ può essere espressa equivalentemente come la differenza tra il valore atteso del quadrato di $X$ e il valore atteso al quadrato di $X$:
	\begin{equation} \label{eq:varalt}
		\var(X) = \ev{X^2}- \ev{X}^2
	\end{equation}
\end{prop}
\begin{proof}
	\begin{align*}
		\var(X) & = \ev{(X-\ev{X})^2}          \bc{definizione \ref{def:var}}   \\
		        & = \ev{X^2-2\mu X+\mu^2}      \bc{sviluppando il quadrato}     \\
		        & = \ev{X^2}-2\mu \ev{X}+\mu^2 \bc{linearità del valore atteso} \\
		        & = \ev{X^2}-2\mu^2+\mu^2      \bc{definizione di $\mu$}        \\
		        & = \ev{x^2}- \ev{x}^2
	\end{align*}
\end{proof}

\begin{prop} \label{prop:indvar}
	La varianza della funzione indicatrice di un evento $A$ è uguale al prodotto delle probabilità di $A$ e del suo complementare:
	\begin{equation*}
		\var(I_A) = P(A)P(\bar A)
	\end{equation*}
\end{prop}
\begin{proof}
	\begin{align*}
		\var(I_A) & = \ev{I_A^2}- \ev{I_A}^2                             \\
		          & = \ev{I_A}- \ev{I_A}^2   \bc{idempotenza di $I_A$}   \\
		          & = \ev{I_A}(1- \ev{I_A})  \bc{proprietà distributiva} \\
		          & = P(A)P(\bar A)
	\end{align*}
\end{proof}

\begin{prop}
	In una trasformazione lineare su una variabile aleatoria $X$, una traslazione non influisce sulla varianza; al contrario, una dilatazione si traduce in una dilatazione quadratica.
	\begin{equation*}
		\var(aX+b)=a^2\var(X)
	\end{equation*}
\end{prop}
\begin{proof}
	\begin{align*}
		\var(aX+b) & = \ev{(aX+b-\ev{aX+b})^2} \\
		           & = \ev{(aX+b-a\ev{X}-b)^2} \\
		           & = \ev{(a(X-\ev{X}))^2}    \\
		           & = a^2~\ev{(X-\ev{X})^2}   \\
		           & = a^2 \var(X)             \\
	\end{align*}
\end{proof}



\subsection{Funzione di ripartizione congiunta}
\begin{defin}
	Data una coppia di variabili aleatorie $X, Y$, la funzione di ripartizione congiunta $F_X:\R\times\R\to[0,1]$ (o di distribuzione cumulativa congiunta) è la funzione che associa a una coppia di valori reali per $x$ e $y$ la probabilità che l'esito di $X$ sia minore o uguale a $x$ e l'esito di $Y$ sia minore o uguale a $y$:
	\begin{equation*}
		\forall x,y\in\R\qquad F_{X,Y}(x,y):=P(X\leq x, Y\leq y) \\
	\end{equation*}
\end{defin}

Il limite per $y$ che tende a $+\infty$ della funzione di ripartizione congiunta di $X$ e $Y$ è uguale alla funzione di ripartizione di $X$. In questo caso si dice che $F_X(x)$ è la distribuzione marginale rispetto a $X$:
\begin{equation} \label{eq:distrmargin}
	\lim_{y\to+\infty}F_{X,Y}(x, y) = \lim_{y\to+\infty}P(X\leq x,Y\leq y) = P(X\leq x) = F_X(x)
\end{equation}
Analogo vale, ovviamente, per $x\to+\infty$.



\subsection{Funzione di massa di probabilità congiunta}
\begin{defin}
	Data una coppia di variabili aleatorie $X, Y$, la funzione di massa di probabilità congiunta $p_{X,Y}:\R\times\R\to[0,1]$ è la funzione che associa a una coppia di valori reali per $x$ e $y$ la probabilità che l'esito di $X$ sia uguale a $x$ e quello di $Y$ sia uguale a $y$:
	\begin{equation*}
		p_{X,Y}(x,y)=P(X=x,Y=y)
	\end{equation*}
\end{defin}

Analogamente a quanto visto nella \eqref{eq:distrmargin} con il limite, la somma per tutte le specificazioni $y_j$ di $Y$ di $p_{X,Y}(x,y_j)$ è uguale alla massa di probabilità marginale rispetto a $X$:
\begin{equation} \label{eq:massamargin}
	\sum_j p_{X,Y}(x,y_j) = p_X(x)
\end{equation}




\subsection{Indipendenza}
\begin{defin}
	Variabili aleatorie $X$ e $Y$ sono indipendenti se e solo se
	\begin{equation*}
		\forall A,B\subseteq\R\qquad P(X\in A, Y\in B)=P(X\in A)P(Y\in B)
	\end{equation*}
\end{defin}

La definizione è equivalente alla seguente proprietà della funzione di massa di probabilità congiunta:
\begin{prop}
	Variabili aleatorie $X$ e $Y$ sono indipendenti se e solo se
	\begin{equation*}
		a,b\in\R\qquad p_{X,Y}(a,b)=p_X(a)p_Y(b)
	\end{equation*}
\end{prop}
\begin{proof}
	L'implicazione verso destra:
	\begin{gather*}
		\forall A,B\subseteq\R\qquad P(X\in A, Y\in B)=P(X\in A)P(Y\in B) \\
		\Rightarrow \\
		a,b\in\R\qquad p_{X,Y}(a,b)=p_X(a)p_Y(b)
	\end{gather*}
	è immediata dal momento che si possono scegliere sottoinsiemi di $\R$ corrispondenti a singoletti contenenti $a$ e $b$ ($A=\{a\}$ e $B=\{b\}$).

	L'implicazione inversa parte dalla ricostruzione di $A$ e $B$ come unioni di singoletti:
	\begin{align*}
		P(X\in A, Y\in B) & = \sum_{a\in A}\sum_{b\in B} p_{X,Y}(a,b)                           \\
		                  & = \sum_{a\in A}\sum_{b\in B} p_X(a)p_Y(b) \bc{ipotesi}              \\
		                  & = \left(\sum_{a\in A}p_X(a)\right)\left(\sum_{b\in B} p_Y(b)\right) \\
		                  & = P(X\in A)P(Y\in B)
	\end{align*}
\end{proof}

Si dimostra anche la seguente equivalenza:
\begin{prop}
	Variabili aleatorie $X$ e $Y$ sono indipendenti se e solo se
	\begin{equation*}
		a,b\in\R\qquad F_{X,Y}(a,b)=F_X(a)F_Y(b)
	\end{equation*}
\end{prop}



\subsection{Multivariati}
I concetti visti per coppie di variabili aleatorie si estendono a vettori aleatori (più comunemente detti variabili aleatorie multivariate, o multivariati) di dimensione arbitraria:


\subsubsection{Funzione di ripartizione}
\begin{defin}
	Date $n$ variabili aleatorie $X_1, \dots, X_n$, la funzione di ripartizione $F_{X_1,\dots,X_n}:\R^n\to[0,1]$ è la funzione che associa a una $n$-upla di valori reali $x_1,\dots,x_n$ la probabilità che l'esito di $X_i$ sia minore o uguale a $x_i$ per ogni $i$ da $1$ a $n$:
	\begin{equation*}
		F_{X_1,\dots,X_n}(x_1,\dots,x_n) = P(X_1\leq x_1,\dots,X_n\leq x_n)
	\end{equation*}
\end{defin}


\subsubsection{Funzione di massa di probabilità}
\begin{defin}
	Date $n$ variabili aleatorie $X_1, \dots, X_n$, la funzione di massa di probabilità $p_{X_1,\dots,X_n}:\R^n\to[0,1]$ è la funzione che associa a una $n$-upla di valori reali $x_1,\dots,x_n$ la probabilità che l'esito di $X_i$ sia uguale a $x_i$ per ogni $i$ da $1$ a $n$:
	\begin{equation*}
		p_{X_1,\dots,X_n}(x_1,\dots,x_n) = P(X_1=x_1,\dots,X_n=x_n)
	\end{equation*}
\end{defin}


\subsubsection{Indipendenza}
\begin{defin}
	Variabili aleatorie $X_1,\dots,X_n$ sono indipendenti se e solo se
	\begin{equation*}
		\forall A_1,\dots,A_n\subseteq\R \qquad P\left(\bigland_{i=1}^n X_i\in A_i\right) = \prod_{i=1}^n P(X_i\in A_i)
	\end{equation*}
\end{defin}


\subsubsection{Valore atteso}
Il valore atteso per una funzione applicata a una coppia di variabili aleatorie discrete coinvolge tutte le combinazioni\footnote{la sommatoria con doppio indice può essere scomposta in due sommatorie, una per ogni indice} di specificazioni delle due variabili:
\begin{equation*}
	\ev{f(x,y)}=\sum_{i,j}f(x_i,y_j)P(X=x_i,Y=y_i)
\end{equation*}

\begin{prop}
	Il valore atteso della somma di due variabili aleatorie discrete è uguale alla somma dei valori attesi delle stesse.
	\begin{equation*}
		\ev{X+Y}=\ev{X}+\ev{Y}
	\end{equation*}
\end{prop}
\begin{proof}
	\begin{align*}
		\ev{X+Y} & = \sum_i\sum_j (x_i+y_j) P(X=x_i,Y=y_j)                               \\
		         & = \sum_i\sum_j x_i P(X=x_i,Y=y_j) + \sum_i\sum_j y_j P(X=x_i,Y=y_j)   \\
		         & = \sum_i x_i \sum_j P(X=x_i,Y=y_j) + \sum_j y_j \sum_i P(X=x_i,Y=y_j)
	\end{align*}
	Le due sommatorie interne sono probabilità marginali:
	\begin{align*}
		 & = \sum_i x_i P(X=x_i) + \sum_j y_i P(Y=y_i) \\
		 & = \ev{X} + \ev{Y}
	\end{align*}
\end{proof}

Generalizzando:
\begin{prop}
	Il valore atteso della somma di variabili aleatorie discrete è uguale alla somma dei valori attesi delle stesse.
	\begin{equation*}
		\ev{\sum_i X_i}=\sum_i \ev{X_i}
	\end{equation*}
\end{prop}

Di conseguenza, se si può esprimere una variabile $X$ come la somma di variabili $X_1,\dots,X_n$, è sufficiente calcolare il valore atteso di ciascuna componente per calcolare quello della variabile iniziale.


\subsubsection{Altre proprietà}
Il seguente risultato è una generalizzazione della varianza, utile a valutare quanto un'approssimazione $c$ di $X$ è buona. La migliore approssimazione è $c=\ev{X}$, in coerenza con l'interpretazione semantica di $\ev{X}$.
\begin{prop}
	\begin{equation*}
		\ev{(X-c)^2}=\var(X)+(\mu-c)^2\geq\var(X)
	\end{equation*}
\end{prop}
\begin{proof}
	Se $\mu=\ev{X}$
	\begin{align*}
		\ev{(X-c)^2} & = \ev{(X-\mu+\mu-c)^2}                                                     \\
		             & = \ev{(X-\mu)^2-2(X-\mu)(\mu-c)+(\mu-c)^2}                                 \\
		             & = \ev{(X-\mu)^2}-\ev{2(X-\mu)(\mu-c)}+(\mu-c)^2                            \\
		             & = \ev{(X-\mu)^2}-2(\mu-c)\underbrace{\ev{X-\mu}}_{=\ev{X}-\mu=0}+(\mu-c)^2 \\
		             & = \ev{(X-\mu)^2}+(\mu-c)^2
	\end{align*}
\end{proof}



\subsection{Covarianza}

\begin{defin}
	Date due variabili aleatorie $X$ e $Y$, la covarianza tra $X$ e $Y$ è così definita:
	\begin{equation} \label{eq:covar}
		\cov(X,Y)=\ev{(X-\mu_X)(Y-\mu_Y)}
	\end{equation}
\end{defin}
\noindent
Ovviamente la covarianza è simmetrica, ossia $\cov(X,Y)=\cov(Y,X)$.


\subsubsection{Proprietà}
La seguente è una definizione equivalente per la covarianza tra due variabili aleatorie:
\begin{prop} \label{prop:covalt}
	Date due variabili aleatorie $X$ e $Y$, la covarianza tra $X$ e $Y$ è uguale al valore atteso del prodotto delle due meno il prodotto dei rispettivi valori attesi:
	\begin{equation*}
		\cov(X,Y)=\ev{XY}-\ev{X}\ev{Y}
	\end{equation*}
\end{prop}
\begin{proof}
	Sviluppando la \eqref{eq:covar}:
	\begin{align*}
		\cov(X,Y) & = \ev{XY-\mu_X Y-\mu_Y x + \mu_X\mu_Y}                                                            \\
		          & = \ev{X,Y}-\underbrace{\mu_X\ev{Y}}_{\mu_X\mu_Y}-\underbrace{\mu_Y\ev{X}}_{\mu_X\mu_Y}+\mu_X\mu_Y \\
		          & =\ev{XY}-\ev{X}\ev{Y}
	\end{align*}
\end{proof}

\begin{prop}
	La covarianza è lineare rispetto alla dilatazione di una variabile per una costante:
	\begin{equation*}
		\cov(aX,Y)=a\cov(X,Y)
	\end{equation*}
\end{prop}
\begin{proof}
	Applicando la proprietà \ref{prop:covalt}:
	\begin{align*}
		\cov(aX,Y) & = \ev{aXY}-\ev{aX}\ev{Y}  \\
		           & = a(\ev{XY}-\ev{X}\ev{Y}) \\
		           & = a\cov(X,Y)
	\end{align*}
\end{proof}

\begin{prop} \label{prop:covsum0}
	\begin{equation*}
		\cov(X+Y,Z)=\cov(X,Z)+\cov(Y,Z)
	\end{equation*}
\end{prop}
\begin{proof}
	\begin{align*}
		\cov(X+Y,Z) & = \ev{(X+Y)Z} - \ev{X+Y}\ev{Z}                    \\
		            & = \ev{XZ+YZ} - (\ev{X} + \ev{Y}) \ev{Z}           \\
		            & = \ev{XZ} + \ev{YZ} - \ev{X}\ev{Z} - \ev{Y}\ev{Z} \\
		            & = \cov(X,Z) + \cov(Y,Z)
	\end{align*}
\end{proof}

Generalizzando:
\begin{prop} \label{prop:covsum}
	\begin{equation*}
		\cov\left(\sum_i X_i,\sum_j Y_j\right)=\sum_i\sum_j\cov(X_i,Y_j)
	\end{equation*}
\end{prop}
\begin{proof}
	\begin{align*}
		\cov\left(\sum_i X_i,\sum_j Y_j\right) & = \sum_i\cov\left(X_i,\sum_j Y_j\right) \bc{proprietà \ref{prop:covsum0}} \\
		                                       & = \sum_i\sum_j \cov(X_i,Y_j) \bc{proprietà \ref{prop:covsum0}}            \\
	\end{align*}
\end{proof}

\begin{prop}
	La covarianza tra una variabile aleatoria $X$ e se stessa è uguale alla varianza di $X$:
	\begin{equation*}
		\cov(X,X)=\var(X)
	\end{equation*}
\end{prop}
\begin{proof}
	\begin{align*}
		\cov(X,X) & = \ev{(X-\mu)(X-\mu)} \\
		          & = \ev{(X-\mu)^2}      \\
		          & = \var(X)
	\end{align*}
\end{proof}

\begin{prop} \label{prop:varsum}
	La varianza della somma di due variabili aleatorie $X$ e $Y$ è uguale alla somma delle rispettive varianze più il doppio della loro covarianza:
	\begin{equation*}
		\var(X+Y) = \var(X)+\var(Y)+2\cov(X,Y)
	\end{equation*}
\end{prop}
\begin{proof}
	Usando la definizione alternativa (proprietà \ref{prop:varalt}):
	\begin{align*}
		\var(X+Y) & = \ev{(X+Y)^2} - \ev{X+Y}^2                                            \\
		          & = \ev{X^2+2XY+Y^2}-(\ev{X}+\ev{Y})^2                                   \\
		          & = \ev{X^2} + 2\ev{XY} + \ev{Y^2} - \ev{X}^2 - 2\ev{X}\ev{Y} - \ev{Y}^2 \\
		          & = \ev{X^2} - \ev{X}^2 + \ev{Y^2} - \ev{Y}^2 + 2(\ev{XY}-\ev{X}\ev{Y})  \\
		          & = \var(X) + \var(Y) + 2\cov(X,Y)
	\end{align*}
\end{proof}

Generalizzando:
\begin{prop} \label{prop:varsumcov}
	\begin{equation*}
		\var\left(\sum_{i=1}^n X_i\right) = \sum_{i=1}^n \var(X_i) + \sum_{i\neq j}\cov(X_i,Y_j)
	\end{equation*}
\end{prop}
\noindent

\begin{teor}
	Il valore atteso del prodotto di variabili aleatorie discrete indipendenti è uguale al prodotto dei rispettivi valori attesi:
	\begin{equation*}
		\ev{XY} = \ev{X}\ev{Y}
	\end{equation*}
\end{teor}
\begin{proof}
	\begin{align*}
		\ev{XY} & = \sum_i\sum_j x_i y_j P(X=x_i,Y=y_i)                              \\
		        & = \sum_i\sum_j x_i y_j P(X=x_i)P(Y=y_j)                            \\
		        & = \left(\sum_i x_i P(X=x_i)\right)\left(\sum_j y_j P(Y=y_j)\right) \\
		        & = \ev{X}\ev{Y}
	\end{align*}
\end{proof}

\begin{corol}
	La covarianza di due variabili aleatorie discrete indipendenti $X$ e $Y$ è nulla:
	\begin{equation*}
		\cov(X,Y) = 0
	\end{equation*}
\end{corol}
\begin{proof}
	\begin{align*}
		\cov(X,Y) & = \ev{XY}-\ev{X}\ev{Y}      \\
		          & = \ev{X}\ev{Y}-\ev{X}\ev{Y} \\
		          & = 0
	\end{align*}
\end{proof}

\begin{corol}
	Date variabili indipendenti $X_1,\dots,X_n$:
	\begin{equation*}
		\var\left(\sum_{i=1}^n X_i\right) = \sum_{i=1}^n \var(X_i)
	\end{equation*}
\end{corol}
\begin{proof}
	Applicando la proprietà \ref{prop:varsum}:
	\begin{align*}
		\var\left(\sum_{i=1}^n X_i\right) & = \sum_{i=1}^n\var(X_i)+\sum_{i\neq j}\cov(X_i,X_j) \\
		                                  & = \sum_{i=1}^n \var(X_i)
	\end{align*}
\end{proof}


\begin{prop}
	Date funzioni indicatrici di eventi, o in generale variabili aleatorie bernoulliane (ossia di specificazione binaria):
	\begin{equation*}
		\cov(X,Y)>0 \Rightarrow P(X=1\mid Y=1)>P(X=1)
	\end{equation*}
\end{prop}
\begin{proof}
	\begin{align*}
		\cov(X,Y) & = \ev{XY} - \ev{X}\ev{Y}  \\
		          & = P(XY=1)-P(X=1)P(Y=1)    \\
		          & = P(X=1,Y=1)-P(X=1)P(Y=1) \\
	\end{align*}
	Se $\cov(X,Y)>0$
	\begin{align*}
		P(X=1,Y=1)                & > P(X=1)P(Y=1) \\
		\frac{P(X=1,Y=1)}{P(Y=1)} & > P(X=1)       \\
		P(X=1\mid Y=1)            & > P(X=1)
	\end{align*}
\end{proof}


\subsubsection{Coefficiente di correlazione}
\begin{defin}
	Il coefficiente di correlazione $\rho$ è un indice di correlazione tra variabili aleatorie discrete ed è così definito:
	\begin{equation*}
		\rho= \frac{\cov(X,Y)}{\sigma_X\sigma_Y}
	\end{equation*}
\end{defin}
